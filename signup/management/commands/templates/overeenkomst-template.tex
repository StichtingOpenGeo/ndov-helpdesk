\documentclass[10pt, a4paper]{article}
\usepackage[a4paper,top=2.0cm, bottom=1.0cm]{geometry}
\usepackage[utf8]{inputenc}
\usepackage[dutch]{babel}
\usepackage{graphicx}
\usepackage{listings}
\renewcommand{\rmdefault}{ppl}
\linespread{1.2}
\setlength{\parindent}{0in}
\title{Leveringsovereenkomst Marktpartij ND-OV\vspace{-2ex}}
\begin{document}
\maketitle
\thispagestyle{empty}
\pagestyle{empty}
\textbf{Stichting OpenGeo}, gevestigd en kantoorhoudende te Leidschendam, te dezer zake rechtsgeldig vertegenwoordigd door \textbf{Stefan de Konink, voorzitter}, hierna te noemen ``loket'';
\begin{center}
en\\
\end{center}
\ifdefined\onderneming
    \textbf{\onderneming} te dezer zake rechtsgeldig vertegenwoordigd door \textbf{\tekenbevoegd, \functie}, hierna te noemen ``afnemer'';
\else
    \textbf{\tekenbevoegd}, hierna te noemen ``afnemer'';
\fi

\section*{Overwegende dat:}
\begin{enumerate}
\item OV autoriteiten willen dat brongegevens van reisinformatie vrij voor hergebruik ter beschikking komen voor afnemers;
\item Brongegevens aan afnemers beschikbaar worden gesteld onder een CC0 vrijwaring;
\item Brongegevens aan afnemers worden verstrekt door een loket;
\item Levering van brongegevens aan een loket geschied op basis van afspraken tussen overheden en vervoerders;
\item Een loket als taak heeft om brongegevens ter beschikking te stellen aan afnemers;
\item Een afnemer een loket kan aanspreken op het tijdig doorleveren van de brongegevens;
\item Middels een gebruiksovereenkomst het gelijke speelveld tussen afnemers is gewaarborgd;
\item Er een leveringszekerheid aan afnemers is in verband met investeringen die afnemers moeten doen;
\item Een leverancier de brongegevens gratis verstrekt aan een loket;
\item Een gebruiker/afnemer die misbruik maakt van brongegevens door de leverancier van de brongegevens aansprakelijk kan worden gesteld wegens smaad;
\item Een loket afnemers (mogelijk) een vergoeding voor verstrekking van brongegevens in rekening kan brengen;
\item Het monitoren van de concessie-uitvoering geschiedt op basis van afspraken tussen de OV-authoriteit (concessieverlener) en de vervoerder (concessiehouder).
\end{enumerate}

\section*{\textsc{Verklaren te zijn overeengekomen:}}
\section{Algemeen}
Een loket levert aan afnemers onbewerkte brongegevens, die door leveranciers worden geleverd.
Een loket kan niet worden aangesproken op gegevens die niet door leveranciers worden geleverd.
\newpage

\section{Definities}

\subsection*{Begrippen:}
\begin{tabular}{l|p{10cm}}
Loket                       & Organisatie die brongegevens reisinformatie openbaar vervoer verzamelt en deze onbewerkt ter beschikking stelt aan afnemers. \\
\hline
Brongegevens                & Gegevens zoals deze gedefinieerd zijn in de (dynamische) bijlage 1 van dit document. \\
\hline
Vergoeding voor leveren     & Een loket kan een vergoeding bij afnemers vragen voor het leveren van brongegevens. Voor brongegevens zelf wordt geen vergoeding gevraagd. \\
\hline
Afnemers                    & Partijen die brongegevens afnemen van een loket en brongegevens zelf gebruiken en/of deze doorleveren aan andere partijen. \\
\hline
Leveranciers                & Vervoerders en decentrale overheden. \\
\end{tabular}

\section{Gebruiksrecht}
\begin{enumerate}
   \item Een loket verleent aan een afnemer, vanaf de datum van ondertekening van deze gebruikersovereenkomst, een niet-exclusieve licentie voor het gebruik van brongegevens. Afnemer mag brongegevens, al dan niet in bewerkte vorm, doorleveren aan partijen onder een CC0 vrijwaring.
   \item Loket verstrekt brongegevens.
   \item Een loket is in staat om wijzigingen in de structuur van een brongegeven te verwerken. Wijzigingen in de structuur van een brongegeven worden minimaal 6 maanden voor invoering van de wijziging door een loket doorgegeven aan de afnemers. Een loket is in staat om uitbreidingen in het assortiment van brongegevens binnen een maand te verwerken.
   \item De beschrijving van brongegevens inclusief een beschrijving van bijbehorende documentatie is te vinden in de bijlage bij deze overeenkomst. Deze bijlage is een dynamisch document. Door overheden en vervoerders zullen / kunnen nieuwe afspraken omtrent brongegevens worden gemaakt.
   \item Spoorbrongegevens zijn niet geschikt voor het interpreteren van Openbaar Vervoer prestaties.
\end{enumerate}

\section{Vergoeding voor verstrekking van brongegevens}
\begin{enumerate}
   \item De verstrekkingskosten worden jaarlijks bepaald.\\ \textit{Peildatum 1 januari 2020: Er wordt \textbf{geen vergoeding} gerekend in dit kalenderjaar.}
   \item Wijzigingen van de verstrekkingskosten worden ten minste drie maanden voor de aanvang van het nieuwe kalenderjaar aangekondigd.
   \item De vergoeding voor verstrekking van brongegevens per afnemer bedraagt ten hoogste 1000 euro per jaar exclusief BTW.
   \item De vergoeding wordt per maand bepaald en gefactureerd naar gebruik, de betalingstermijn is 30 dagen na ontvangst van de factuur.
   \item Een loket is gerechtigd de vergoeding voor onbepaalde tijd op te schorten, wanneer:
   \begin{enumerate}
        \item De verstrekkingskosten nihil zijn.
        \item Door afnemer per week minimaal 0.1fte wordt bijgedragen aan de uitvoering van het loket.
   \end{enumerate}
\end{enumerate}

\section{Looptijd, opzegging en beëindiging}
\subsection*{Looptijd}
\begin{enumerate}
   \item De duur van deze gebruikersovereenkomst is onbepaald.
\end{enumerate}

\subsection*{Opzegging}
\begin{enumerate}
   \item Afnemer kan deze gebruikersovereenkomst op ieder moment schriftelijk opzeggen.
   \item Een loket kan deze gebruikersovereenkomst schriftelijk opzeggen met een opzegtermijn van drie jaar.
\end{enumerate}

\subsection*{Beëindiging}
\begin{enumerate}
   \item Een loket kan deze gebruikersovereenkomst met onmiddellijke ingang beëindigen door middel van een schriftelijke verklaring, indien afnemer ook na schriftelijke aanmaning, binnen een redelijke termijn, in strijd handelt met één van de artikelen van deze gebruikssovereenkomst.
   \item Een loket kan, zonder dat ter zake ingebrekestelling zal zijn vereist, de gebruiksovereenkomst zonder gerechtelijke tussenkomst beëindigen, indien:
      \begin{enumerate}
      \item Afnemer (voorlopige) surseance van betaling aanvraagt of hem (voorlopige) surseance van betaling wordt verleend;
      \item Afnemer faillissement aanvraagt of in staat van faillissement wordt verklaard;
      \item De onderneming van Afnemer wordt geliquideerd;
      \item Afnemer zijn huidige onderneming staakt;
      \item Op (een aanmerkelijk deel van) het vermogen van Afnemer beslag wordt gelegd en het beslag niet binnen een maand is opgeheven;
      \item Afnemer anderszins niet langer in staat moet worden geacht de verplichtingen uit deze gebruiksovereenkomst na te kunnen komen.
      \end{enumerate}
\end{enumerate}
\newpage

\section{Aansprakelijkheid}
\begin{enumerate}
   \item Een loket sluit elke aansprakelijkheid uit.
\end{enumerate}

\section{Functiebeschrijving loket}
\begin{enumerate}
   \item Voor het melden van incidenten (afwijking van de normale situatie) is een loket continu bereikbaar per e-mail. Incidenten kunnen telefonisch tussen tenminste 08.00 uur en 22.00 worden doorgegeven. Incidenten worden opgenomen in een transparant incident informatiesysteem. Binnen vier uur zal een reactie worden gegeven.
   \item Vragen en verzoeken kunnen tijdens kantooruren worden ingediend.
   \item Een loket is aanspreekbaar op tijdigheid van doorleveren van realtime brongegevens binnen 3 seconden en een loket is aanspreekbaar op een leveringszekerheid (tijd dat het systeem werkt) van tenminste 99\%.
   \item Een loket kondigt onderhoud aan systemen vooraf aan.
   \item Een loket en afnemer communiceren transparant over storingen aan systemen.
   \item Leveranciers zijn via een loket aanspreekbaar op de kwaliteit van brongegevens.
\end{enumerate}

\section{Geschilbeslechting en toepassingsrecht}
Op deze gebruikersovereenkomst is het Nederlandse recht van toepassing.
Geschillen die mochten ontstaan naar aanleiding van deze gebruiksovereenkomst, zullen worden beslecht door een bevoegde rechter te Den Haag.

\section*{}
In tweevoud ondertekend op \today, te \vestigingsplaats.\\
\\
\begin{tabular}{l p{5cm} l}
Namens afnemer: & & Namens loket:\\
\vspace{3cm}
\\
\tekenbevoegd & & Stefan de Konink \\
\ifdefined\onderneming\onderneming\fi & & Stichting OpenGeo
\end{tabular}

\end{document}
